\documentclass[10pt,letterpaper,twoside,openright]{article}
\usepackage[utf8]{inputenc}
\usepackage{amsmath}
\usepackage{amsfonts}
\usepackage{amssymb}

\author{Carlos Antonio Macías Duarte}
\title{Lista de Algoritmos Heurísticos}
\usepackage[spanish, es-tabla]{babel}
\usepackage[vlined, spanish, onelanguage, ruled, linesnumbered, lined]{algorithm2e}

\begin{document}

	\maketitle
		
	\tableofcontents
	
	\section{Recocido Simulado}
	
		\begin{algorithm}[H]
			\caption{Recocido Simulado (SA)}
			\KwIn{\\$ n $: Tamaño de vecindad\\ $ c $: Control de temperatura\\$ t $: Número de iteraciones}
		 	\KwOut{\\$ g $ : Mejor solución}
				$ i \leftarrow 1 $ : Contador de iteraciones\;
				$ g \leftarrow generarSolucionInicial() $ : Solución inicial\;
				\While {$ i \leq t $}
				{
					$ p \leftarrow \emph{generarVecindad(n,g)}$ : Generar nueva vecindad\;
					$ j \leftarrow 1 $ : Contador de la vecindad\;
					\While {$ j \leq n $}
					{
						\uIf {$p_{j}$ mejor que $g$}
						{
							$ g \leftarrow p_{j} $\;
						}
						\uElseIf{$exp(\dfrac{f(g) - f(p_{j})}{c}) > rand(0,1)$}
						{
							$ g \leftarrow p_{j} $\;
						}
						
					}			
					$ i \leftarrow  i + 1 $\;
					$ c \leftarrow  generarNuevoTamanoVecindad() $ : Genera el nuevo tamaño de la vecindad siguiente\;
					$ n \leftarrow  generarNuevaTemperatura() $\ :  Genera el nuevo valor de control de temperatura;
				}
		 		\Return $ g $
		 \end{algorithm}
	
	\section{Algoritmo Genético}
	
	\section{Optimización por Enjambre de Partículas}
	
		\begin{algorithm}[H]
			\caption{Optimización por Enjambre de Partículas (PSO)}
			\KwIn{\\$ n $: Tamaño del enjambre\\ $ t $: Número de iteraciones}
		 	\KwOut{\\$ g $ : Partícula con la mejor posición}
				$ p \leftarrow \emph{inicializaEnjambre(n)}$ : Inicialización del enjambre\;
				$ i \leftarrow 1 $ : Contador de iteraciones\;
				$ g \leftarrow mejorParticula(p) $ : Mejor partícula\;
				\While {$ i \leq t $}
				{
					$ nuevaVelocidad(p,g) $ : Actualiza la velocidad de las partículas\;
					$ nuevaPosicion(p) $ : Calcular nuevas posiciones\;
					$ g \leftarrow mejorParticula(p) $ : Obtener la Mejor partícula\;
					$ i \leftarrow  i + 1 $\;
				}
		 		\Return $ g $
		 \end{algorithm}

\end{document}